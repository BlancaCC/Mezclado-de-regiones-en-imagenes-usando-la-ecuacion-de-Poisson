\documentclass[11pt,a4paper]{article}
\usepackage{blindtext}
\usepackage[T1]{fontenc}
\usepackage[utf8]{inputenc}
\usepackage[english,spanish]{babel}
\usepackage{mathtools}

\begin{document}

\section{Solución discreta}

El problema de variaciones (TODO añadir referencia) asociado a la 
ecuación de Poisson con condiciones de frontera de Dirichlet puede ser discretizado como sigue: 

Sin pérdida de generalidad ahora $S$ y $\Omega$ comienzan a ser conjuntos finitos de puntos definidos
sobre la rejilla discreta. 

Para cada pixel $p$ de $S$, se define $N_p$ el conjunto de a lo sumo cuatro puntos contiguos
vertical y horizontalmente a $p$ en $S$. 
Sea  $\langle p,q\rangle$ el par de puntos cumpliendo que $q$ pertenece a $N_p$.   

Se define el borde de $\Omega$ como 

\begin{equation*}
    \partial\Omega = \{ p \in S \setminus \Omega : N_p \cap \Omega \neq  \emptyset \}
\end{equation*}

Denotaremos como $f_p$ al valor de $f$ en $p$. El objetivo a resolver de nuestro problema es 
calcular el conjunto de intensidades $f_{|_{\Omega}} = \{ f_p, p \in \Omega\}$

Para las condiciones de borde de Dirichlet en el borde de forma arbitraria es mejor (TODO ¿por qué?)
discretizar el problema de variaciones antes que la ecuación de Poisson (TODO: añadir referencias a las ecuaciones). 

Para la fórmula discretizada del problema de variación (TODO referencia) resulta: 

\begin{equation}\label{eq:discretizada}
    \min: { f|\Omega} = \sum_{ \langle p,q\rangle \cap \Omega = \emptyset} (f_p - f_q - v_{pq})^2 \text{ con } f_p = f^*_p \text{ para todo } p \in \partial \Omega
\end{equation}

donde $v_{pq}$ es la proyección de $v(\frac{p+q}{2})$ orientada en el segmento 
$\left[ p,q \right]$, 
esto es $v_{pq} = v(\frac{p+q}{2})\cdot  \overrightarrow{pq}$. 

Este problema \ref{eq:discretizada} pertenece a un problema de optimización cuadrática. 
Una solución de \ref{eq:discretizada} satisface además que 

(TODO: fórmula explicar de dónde viene la fórmula ). 

\begin{equation}\label{eq:ecuacionLinealDiscretizada}
    \text{Para todo } p \in \Omega, |N_p| f_p - \sum_{q \in N_p \cap \Omega} f_p = \sum_{q \in N_p \cap \partial \Omega} f^*_p + \sum_{q \in N_p } v_{pq}.
\end{equation}

La ecuación  \ref{eq:ecuacionLinealDiscretizada} forma un sistema clásico 
%(TODO: Las propiedad habría que traducirlas bien, son las siguientes Equations (7) form a classical, sparse (banded), symmetric, positive-definite system. Because of the arbitrary shape of bound- ary ∂ Ω, we must use well-known iterative solvers.)



Nota: 

El método para resolver el sistema que propone es el de Gauss Seidel. 
http://blog.espol.edu.ec/analisisnumerico/3-6-1-gauss-seidel-ejemplo01/

Podríamos usar alguno de numpy que seguro que va más rápido que el que planteemos nosotros. 

\end{document}